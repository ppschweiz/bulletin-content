\documentclass[11pt, a4paper]{scrartcl}
\usepackage[top=1cm, bottom=1cm, left=1cm, right=1cm]{geometry}
\usepackage{fontspec}
\usepackage{xltxtra}
\usepackage{xunicode}
\usepackage{datatool}
\usepackage{textpos}
\usepackage{eso-pic}
\usepackage{framed}
\usepackage[german=swiss]{csquotes}
\usepackage[english,french,italian,ngerman]{babel}
\usepackage{setspace}

\setromanfont[BoldFont=Aller Bold]{Aller Light}
\setsansfont[BoldFont=Aller Bold]{Aller Light}

\DTLsetseparator{;}
\DTLloaddb[noheader,keys={id,sn,gn,mail,c,st,plz,loc,sal,cd,v}]{people}{people.csv}

\begin{document}

\selectlanguage{french}

\DTLforeach{people}{\id=id,\surname=sn,\givenname=gn,\mail=mail,\country=c,\street=st,\postalcode=plz,\location=loc,\salutation=sal,\code=cd}{%

%\AddToShipoutPicture{\put(0,0){\includegraphics[width=\paperwidth]{half.pdf}}}
%\AddToShipoutPicture{\includegraphics[width=\paperwidth]{half.pdf}}

\begin{minipage}[t][13.42cm][t]{\textwidth}

\begin{textblock}{9}(8.1,2.6)
\givenname~\surname     \\
\street                 \\
\postalcode~\location   \\
\ifstr{\country}{Switzerland}{%
\~                      \\
}{%
\country ~              \\
}%
\end{textblock}

\begin{textblock}{6}(7.5,0)
\includegraphics[width=7.5cm]{ww.pdf}
\end{textblock}

\begin{textblock}{13.5}(0,7.05)
\centering
\scriptsize
plier ici \hspace{3.5cm} plier ici \hspace{3.5cm} plier ici
\end{textblock}

{\LARGE
\textbf{Matériel de vote} pour la \\
\textbf{Votation de base} du \\
Parti Pirate Suisse \\
le 19~Juin~2017

}

\vspace{5.5cm}

\begin{textblock}{13.5}(0,0)
Pour voter procédez comme suit:
\begin{enumerate}
\setlength{\parskip}{0pt}
\setlength{\itemsep}{1pt}
\item Veuillez signer la carte d'électeur.
\item Remplissez le bulletin de vote selon vos préférences.
\item Glissez le bulletin de vote dans une enveloppe et fermez la soigneusement.
\item Glissez cette enveloppe contenant le bulletin ainsi que la carte d'électeur ensemble dans une deuxième enveloppe (plus grande) et refermez bien celle-ci également.
\item Envoyez cette enveloppe complète à l'adresse indiquée sur la carte d'électeur jusqu'au \textbf{19~Juin~2017}.
\end{enumerate}
\end{textblock}

\end{minipage}

\line(1,0){500}
\vspace{1cm}

\begin{minipage}[t][12.5cm][t]{\textwidth}

\begin{textblock}{9}(8.1,2.6)
\underline{\textsuperscript*{ \givenname~\surname, \street, \postalcode~\location }} \\
\vspace{-0.3cm} \\
Piratenpartei Schweiz \\
Präsidium der Piratenversammlung \\
8000 Zürich
\end{textblock}

\begin{textblock}{6}(7.5,0)
\includegraphics[width=7.5cm]{ww.pdf}
\end{textblock}

{\LARGE\textbf{Carte d'électeur}} \\
pour la votation de base du 19~Juin~2017 \\

numéro de membre \textbf{\id}   \\
\givenname~\surname         \\
\street                     \\
\postalcode~\location       \\
\ifstr{\country}{Switzerland}{%
\~                          \\
}{%
\country ~                  \\
}%

\begin{textblock}{6}(0,-0.2)
Vérification:
\end{textblock}

\begin{textblock}{4}(2,-0.2)
\raggedright
\code
\end{textblock}

\vspace{4cm}

\begin{textblock}{6}(0,-0.2)
\begin{framed}
Signature de l'électeur \\
\vspace{1.5cm} ~ \\
\end{framed}
\end{textblock}

\begin{textblock}{6}(7.3,-0.2)
\begin{framed}
Dépouillement \hfill Laissez vide! \\
\vspace{1.5cm} ~ \\
\end{framed}
\end{textblock}


\end{minipage}

\newpage

\begin{minipage}[t][12.5cm][t]{\textwidth}

\begin{textblock}{6}(7.5,0)
\includegraphics[width=7.5cm]{ww.pdf}
\end{textblock}

{\LARGE\textbf{Bulletin de vote}} \\
pour la votation de base du 19~Juin~2017 \\

\vspace{2cm}


\textbf{Élection des réviseurs}

Les votes valables sont \enquote{oui}, \enquote{non} et \enquote{abstention}.

Chaque candidat qui atteint la majorité absolue des vote \enquote{oui} est élu.

\vspace{1cm}
\begin{tabbing}
~\hspace{2cm} \= ~\hspace{8cm} \= ~\hspace{2cm} \kill
~\> Pat Mächler:        \> \_\_\_\_\_\_\_\_\_\_\_\_ \\
\end{tabbing}

\vspace{0.1cm}
\begin{tabbing}
~\hspace{2cm} \= ~\hspace{8cm} \= ~\hspace{2cm} \kill
~\> Andreas Zimmermann: \> \_\_\_\_\_\_\_\_\_\_\_\_ \\
\end{tabbing}

\vspace{0.5cm}
Ne pas marquer le bulletin de vote, s'il vous plaît!

\end{minipage}


}

\end{document}

