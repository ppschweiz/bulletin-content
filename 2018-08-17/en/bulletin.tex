\documentclass[11pt, a4paper]{scrartcl}
\usepackage[top=1cm, bottom=1cm, left=1cm, right=1cm]{geometry}
\usepackage{fontspec}
\usepackage{xltxtra}
\usepackage{xunicode}
\usepackage{datatool}
\usepackage{textpos}
\usepackage{eso-pic}
\usepackage{framed}
\usepackage[german=swiss]{csquotes}
\usepackage[english,french,italian,ngerman]{babel}
\usepackage{setspace}
\usepackage{array}

\setromanfont[BoldFont=Aller Bold]{Aller Light}
\setsansfont[BoldFont=Aller Bold]{Aller Light}

\newcommand{\votingdate}{17.~August~2018}
\newcommand{\ja}{\enquote{\textit{Ja}}}
\newcommand{\nein}{\enquote{\textit{Nein}}}
\newcommand{\enthaltung}{\enquote{\textit{Enthaltung}}}

\DTLsetseparator{;}
\DTLloaddb[noheader,keys={id,sn,gn,mail,c,st,plz,loc,sal,cd,v}]{people}{people.csv}

\begin{document}

\selectlanguage{ngerman}

\DTLforeach{people}{\id=id,\surname=sn,\givenname=gn,\mail=mail,\country=c,\street=st,\postalcode=plz,\location=loc,\salutation=sal,\code=cd}{%

%\AddToShipoutPicture{\put(0,0){\includegraphics[width=\paperwidth]{half.pdf}}}
%\AddToShipoutPicture{\includegraphics[width=\paperwidth]{half.pdf}}

\begin{minipage}[t][13.42cm][t]{\textwidth}

\begin{textblock}{9}(8.1,2.6)
\givenname~\surname     \\
\street                 \\
\postalcode~\location   \\
\ifstr{\country}{Switzerland}{%
\~                      \\
}{%
\country ~              \\
}%
\end{textblock}

\begin{textblock}{6}(7.5,0)
\includegraphics[width=7.5cm]{ww.pdf}
\end{textblock}

\begin{textblock}{13.5}(0,7.05)
\centering
\scriptsize
hier falten \hspace{3.5cm} hier falten \hspace{3.5cm} hier falten
\end{textblock}

{\LARGE
\textbf{Stimmmaterial} für die \\
\textbf{Urabstimmung} der \\
Piratenpartei Schweiz \\
vom \votingdate{}

}

\vspace{5.5cm}

\begin{textblock}{13.5}(0,0)
Zum Abstimmen wie folgt vorgehen:
\begin{enumerate}
\setlength{\parskip}{0pt}
\setlength{\itemsep}{1pt}
\item Stimmrechtsausweis unterschreiben und falten.
\item Stimmzettel gemäss deinen Präferenzen ausfüllen.
\item Stimmzettel in ein kleines Couvert legen und dieses sicher verschliessen.
\item Das verschlossene Stimmcouvert und den Stimmrechtsausweis zusammen in ein zweites (grösseres) Couvert geben und ebenfalls sicher verschliessen.
\item Das vollständige Couvert bitte bis am \textbf{\votingdate{}} an die aufgedruckte Adresse senden.
\end{enumerate}
\end{textblock}

\end{minipage}

\line(1,0){500}
\vspace{1cm}

\begin{minipage}[t][12.5cm][t]{\textwidth}

\begin{textblock}{9}(8.1,2.6)
\underline{\textsuperscript*{ \givenname~\surname, \street, \postalcode~\location }} \\
\vspace{-0.3cm} \\
Piratenpartei Schweiz \\
Präsidium der Piratenversammlung \\
8000 Zürich
\end{textblock}

\begin{textblock}{6}(7.5,0)
\includegraphics[width=7.5cm]{ww.pdf}
\end{textblock}

{\LARGE\textbf{Stimmrechtsausweis}} \\
für die Urabstimmung vom \votingdate{} \\

Mitglied Nr. \textbf{\id}   \\
\givenname~\surname         \\
\street                     \\
\postalcode~\location       \\
\ifstr{\country}{Switzerland}{%
\~                          \\
}{%
\country ~                  \\
}%

\begin{textblock}{6}(0,-0.2)
Verifikation:
\end{textblock}

\begin{textblock}{4}(2,-0.2)
\raggedright
\code
\end{textblock}

\vspace{4cm}

\begin{textblock}{6}(0,-0.2)
\begin{framed}
Unterschrift des Stimmberechtigten \\
\vspace{1.5cm} ~ \\
\end{framed}
\end{textblock}

\begin{textblock}{6}(7.3,-0.2)
\begin{framed}
Auszählung \hfill Leer lassen! \\
\vspace{1.5cm} ~ \\
\end{framed}
\end{textblock}


\end{minipage}

\newpage

\begin{minipage}[t][12.5cm][t]{17.7cm}

\begin{textblock}{6}(7.5,0)
\includegraphics[width=7.5cm]{ww.pdf}
\end{textblock}

{\LARGE\textbf{Stimmzettel 1}} \\
für die Urabstimmung vom \votingdate{} \\

\vspace{2cm}

% ab hier kopieren
\textbf{Parolenfassung zur Volksinitiative \enquote{Ernährungssouveränität}}

\vspace{0.2cm}
Gültige Stimmen sind jeweils \ja{}, \nein{} und \enthaltung{}.

\vspace{0.2cm}
Eine Parole ist gemäss Artikel~6 Absatz~3 des Versammlungs- und Abstimmungsreglements in zwei Abstimmungen zu fassen. In den Übergangsbestimmungen in Artikel~B Absatz~3bis ist festgesetzt, dass bei brieflicher Urabstimmung die beiden Abstimmnungen gleichzeitig erfolgen.

\vspace{0.2cm}
In der Abstimmung~1 stimmst du gemäss deiner Präferenz zur Vorlage. Das Ergebnis von Abstimmung~1 ergibt sich daraus, ob es mehr \textit{Ja}-Stimmen oder \textit{Nein}-Stimmen hat. 

\vspace{0.2cm}
In der Abstimmung~2 stimmst du darüber ab, ob die Piratenpartei Schweiz die \textit{Ja}-Parole beziehungsweise \textit{Nein}-Parole gemäss Abstimmung~1 publizieren soll. Wenn du bei der Abstimmung~2 auf jeden Fall für eine Stimmfreigabe stimmen möchtest, stimme zwei Mal \nein{}. Wenn du die Parole in jedem Fall publiziert haben möchtest, stimme zwei Mal \ja{}.

\vspace{1cm}

\textbf{Abstimmung 1}

\vspace{0.5cm}

\begin{tabular}{ b{13.5cm} b{3cm} }
Nimmst du die Volksinitiative \enquote{Für Ernährungssouveränität. Die Landwirtschaft betrifft uns alle} an?
& \_\_\_\_\_\_\_\_\_\_\_\_\_\_\_\_ \\
\end{tabular}

\vspace{0.5cm}

\textbf{Abstimmung 2}

\vspace{0.5cm}

\begin{tabular}{ b{13.5cm} b{3cm} }
Soll die Piratenpartei Schweiz die in Abstimmung 1 gefasste Parole bei einem \ja{} vertreten?
& \_\_\_\_\_\_\_\_\_\_\_\_\_\_\_\_ \\
\end{tabular}

\vspace{0.5cm}

\begin{tabular}{ b{13.5cm} b{3cm} }
Soll die Piratenpartei Schweiz die in Abstimmung 1 gefasste Parole bei einem \nein{} vertreten? 
& \_\_\_\_\_\_\_\_\_\_\_\_\_\_\_\_ \\
\end{tabular}

\vspace{0.5cm}

Den Stimmzettel bitte nicht markieren!

\end{minipage}

\newpage

\begin{minipage}[t][12.5cm][t]{17.7cm}

\begin{textblock}{6}(7.5,0)
\includegraphics[width=7.5cm]{ww.pdf}
\end{textblock}

{\LARGE\textbf{Stimmzettel 2}} \\
für die Urabstimmung vom \votingdate{} \\

\vspace{2cm}

% ab hier kopieren
\textbf{Parolenfassung zur Volksinitiative \enquote{Fair-Food}}

\vspace{0.2cm}
Gültige Stimmen sind jeweils \ja{}, \nein{} und \enthaltung{}.

\vspace{0.2cm}
Eine Parole ist gemäss Artikel~6 Absatz~3 des Versammlungs- und Abstimmungsreglements in zwei Abstimmungen zu fassen. In den Übergangsbestimmungen in Artikel~B Absatz~3bis ist festgesetzt, dass bei brieflicher Urabstimmung die beiden Abstimmnungen gleichzeitig erfolgen.

\vspace{0.2cm}
In der Abstimmung~1 stimmst du gemäss deiner Präferenz zur Vorlage. Das Ergebnis von Abstimmung~1 ergibt sich daraus, ob es mehr \textit{Ja}-Stimmen oder \textit{Nein}-Stimmen hat. 

\vspace{0.2cm}
In der Abstimmung~2 stimmst du darüber ab, ob die Piratenpartei Schweiz die \textit{Ja}-Parole beziehungsweise \textit{Nein}-Parole gemäss Abstimmung~1 publizieren soll. Wenn du bei der Abstimmung~2 auf jeden Fall für eine Stimmfreigabe stimmen möchtest, stimme zwei Mal \nein{}. Wenn du die Parole in jedem Fall publiziert haben möchtest, stimme zwei Mal \ja{}.

\vspace{1cm}

\textbf{Abstimmung 1}

\vspace{0.5cm}

\begin{tabular}{ b{13.5cm} b{3cm} }
Nimmst du die Volksinitiative \enquote{Für gesunde sowie umweltfreundlich und fair hergestellte Lebensmittel (Fair-Food-Initiative)}. an?
& \_\_\_\_\_\_\_\_\_\_\_\_\_\_\_\_ \\
\end{tabular}

\vspace{0.5cm}

\textbf{Abstimmung 2}

\vspace{0.5cm}

\begin{tabular}{ b{13.5cm} b{3cm} }
Soll die Piratenpartei Schweiz die in Abstimmung 1 gefasste Parole bei einem \ja{} vertreten?
& \_\_\_\_\_\_\_\_\_\_\_\_\_\_\_\_ \\
\end{tabular}

\vspace{0.5cm}

\begin{tabular}{ b{13.5cm} b{3cm} }
Soll die Piratenpartei Schweiz die in Abstimmung 1 gefasste Parole bei einem \nein{} vertreten? 
& \_\_\_\_\_\_\_\_\_\_\_\_\_\_\_\_ \\
\end{tabular}

\vspace{0.5cm}

Den Stimmzettel bitte nicht markieren!

\end{minipage}

\newpage

\begin{minipage}[t][12.5cm][t]{17.7cm}

\begin{textblock}{6}(7.5,0)
\includegraphics[width=7.5cm]{ww.pdf}
\end{textblock}

{\LARGE\textbf{Stimmzettel 3}} \\
für die Urabstimmung vom \votingdate{} \\

\vspace{2cm}

% ab hier kopieren
\textbf{Parolenfassung zum Gegenvorschlag zur \enquote{Velo-Initiative}}

\vspace{0.2cm}
Gültige Stimmen sind jeweils \ja{}, \nein{} und \enthaltung{}.

\vspace{0.2cm}
Eine Parole ist gemäss Artikel~6 Absatz~3 des Versammlungs- und Abstimmungsreglements in zwei Abstimmungen zu fassen. In den Übergangsbestimmungen in Artikel~B Absatz~3bis ist festgesetzt, dass bei brieflicher Urabstimmung die beiden Abstimmnungen gleichzeitig erfolgen.

\vspace{0.2cm}
In der Abstimmung~1 stimmst du gemäss deiner Präferenz zur Vorlage. Das Ergebnis von Abstimmung~1 ergibt sich daraus, ob es mehr \textit{Ja}-Stimmen oder \textit{Nein}-Stimmen hat. 

\vspace{0.2cm}
In der Abstimmung~2 stimmst du darüber ab, ob die Piratenpartei Schweiz die \textit{Ja}-Parole beziehungsweise \textit{Nein}-Parole gemäss Abstimmung~1 publizieren soll. Wenn du bei der Abstimmung~2 auf jeden Fall für eine Stimmfreigabe stimmen möchtest, stimme zwei Mal \nein{}. Wenn du die Parole in jedem Fall publiziert haben möchtest, stimme zwei Mal \ja{}.

\vspace{1cm}

\textbf{Abstimmung 1}

\vspace{0.5cm}

\begin{tabular}{ b{13.5cm} b{3cm} }
Nimmst du den \enquote{Bundesbeschluss über die Velowege sowie die Fuss- und Wanderwege (direkter Gegenentwurf zur zurückgezogenen Volksinitiative \enquote{Zur Förderung der Velo-, Fuss- und Wanderwege (Velo-Initiative)}).} an?
& \_\_\_\_\_\_\_\_\_\_\_\_\_\_\_\_ \\
\end{tabular}

\vspace{0.5cm}

\textbf{Abstimmung 2}

\vspace{0.5cm}

\begin{tabular}{ b{13.5cm} b{3cm} }
Soll die Piratenpartei Schweiz die in Abstimmung 1 gefasste Parole bei einem \ja{} vertreten?
& \_\_\_\_\_\_\_\_\_\_\_\_\_\_\_\_ \\
\end{tabular}

\vspace{0.5cm}

\begin{tabular}{ b{13.5cm} b{3cm} }
Soll die Piratenpartei Schweiz die in Abstimmung 1 gefasste Parole bei einem \nein{} vertreten? 
& \_\_\_\_\_\_\_\_\_\_\_\_\_\_\_\_ \\
\end{tabular}

\vspace{0.5cm}

Den Stimmzettel bitte nicht markieren!

\end{minipage}
}

\end{document}

