\documentclass[11pt, a4paper]{scrartcl}
\usepackage[top=1cm, bottom=1cm, left=1cm, right=1cm]{geometry}
\usepackage{fontspec}
\usepackage{xltxtra}
\usepackage{xunicode}
\usepackage{datatool}
\usepackage{textpos}
\usepackage{eso-pic}
\usepackage{framed}
\usepackage[german=swiss]{csquotes}
\usepackage[english,french,italian,ngerman]{babel}
\usepackage{setspace}
\usepackage{array}

\setromanfont[BoldFont=Aller Bold]{Aller Light}
\setsansfont[BoldFont=Aller Bold]{Aller Light}

\DTLsetseparator{;}
\DTLloaddb[noheader,keys={id,sn,gn,mail,c,st,plz,loc,sal,cd,v}]{people}{people.csv}

\begin{document}

\selectlanguage{ngerman}

\DTLforeach{people}{\id=id,\surname=sn,\givenname=gn,\mail=mail,\country=c,\street=st,\postalcode=plz,\location=loc,\salutation=sal,\code=cd}{%

%\AddToShipoutPicture{\put(0,0){\includegraphics[width=\paperwidth]{half.pdf}}}
%\AddToShipoutPicture{\includegraphics[width=\paperwidth]{half.pdf}}

\begin{minipage}[t][13.42cm][t]{\textwidth}

\begin{textblock}{9}(8.1,2.6)
\givenname~\surname     \\
\street                 \\
\postalcode~\location   \\
\ifstr{\country}{Switzerland}{%
\~                      \\
}{%
\country ~              \\
}%
\end{textblock}

\begin{textblock}{6}(7.5,0)
\includegraphics[width=7.5cm]{ww.pdf}
\end{textblock}

\begin{textblock}{13.5}(0,7.05)
\centering
\scriptsize
hier falten \hspace{3.5cm} hier falten \hspace{3.5cm} hier falten
\end{textblock}

{\LARGE
\textbf{Stimmmaterial} für die \\
\textbf{Urabstimmung} der \\
Piratenpartei Schweiz \\
vom 17.~Januar~2018

}

\vspace{5.5cm}

\begin{textblock}{13.5}(0,0)
Zum Abstimmen wie folgt vorgehen:
\begin{enumerate}
\setlength{\parskip}{0pt}
\setlength{\itemsep}{1pt}
\item Stimmrechtsausweis unterschreiben und falten.
\item Stimmzettel gemäss deinen Präferenzen ausfüllen.
\item Stimmzettel in ein kleines Couvert legen und dieses sicher verschliessen.
\item Das verschlossene Stimmcouvert und den Stimmrechtsausweis zusammen in ein zweites (grösseres) Couvert geben und ebenfalls sicher verschliessen.
\item Das vollständige Couvert bitte bis am \textbf{17.~Januar~2018} an die aufgedruckte Adresse senden.
\end{enumerate}
\end{textblock}

\end{minipage}

\line(1,0){500}
\vspace{1cm}

\begin{minipage}[t][12.5cm][t]{\textwidth}

\begin{textblock}{9}(8.1,2.6)
\underline{\textsuperscript*{ \givenname~\surname, \street, \postalcode~\location }} \\
\vspace{-0.3cm} \\
Piratenpartei Schweiz \\
Präsidium der Piratenversammlung \\
8000 Zürich
\end{textblock}

\begin{textblock}{6}(7.5,0)
\includegraphics[width=7.5cm]{ww.pdf}
\end{textblock}

{\LARGE\textbf{Stimmrechtsausweis}} \\
für die Urabstimmung vom 17.~Januar~2018 \\

Mitglied Nr. \textbf{\id}   \\
\givenname~\surname         \\
\street                     \\
\postalcode~\location       \\
\ifstr{\country}{Switzerland}{%
\~                          \\
}{%
\country ~                  \\
}%

\begin{textblock}{6}(0,-0.2)
Verifikation:
\end{textblock}

\begin{textblock}{4}(2,-0.2)
\raggedright
\code
\end{textblock}

\vspace{4cm}

\begin{textblock}{6}(0,-0.2)
\begin{framed}
Unterschrift des Stimmberechtigten \\
\vspace{1.5cm} ~ \\
\end{framed}
\end{textblock}

\begin{textblock}{6}(7.3,-0.2)
\begin{framed}
Auszählung \hfill Leer lassen! \\
\vspace{1.5cm} ~ \\
\end{framed}
\end{textblock}


\end{minipage}

\newpage

\begin{minipage}[t][12.5cm][t]{17.7cm}

\begin{textblock}{6}(7.5,0)
\includegraphics[width=7.5cm]{ww.pdf}
\end{textblock}

{\LARGE\textbf{Stimmzettel}} \\
für die Urabstimmung vom 17.~Januar~2018 \\

\vspace{2cm}

% ab hier kopieren
\textbf{Parolenfassung zur Volksinitiative \enquote{No-Billag}}

\vspace{0.1cm}
Gültige Stimmen sind jeweils \enquote{Ja}, \enquote{Nein} und \enquote{Enthaltung}.

\vspace{0.1cm}
Eine Parolenfassung hat gemäss Artikel 6 Absatz 3 des Versammlungs- und Abstimmungsreglement in zwei Abstimmungen zu fassen. In den Übergangsbestimmungen in Artikel B Absatz 3 ist festgesetzt, dass bei brieflicher Urabstimmung die beiden Abstimmnungen gleichzeitig per Wahl durch Zustimmung erfolgen.

\vspace{0.1cm}
Ihr dürft also bei jeder Abstimmungsfrage mit \enquote{Ja}, \enquote{Nein} oder \enquote{Enthaltung} antworten. Gewählt ist pro Abstimmung jene Option, die eine einfache Mehrheit von \enquote{Ja}-Stimmen auf sich vereinigt. 

\vspace{1cm}

\textbf{Abstimmung 1}

\vspace{0.5cm}

\begin{tabular}{ b{13.5cm} b{3cm} }
Nimmst du die No-Billag Initiative "Ja zur Abschaffung der Radio- und Fernsehgebühren (Abschaffung der Billag-Gebühren)" an?
& \_\_\_\_\_\_\_\_\_\_\_\_\_\_\_\_ \\
\end{tabular}

\vspace{0.5cm}

\textbf{Abstimmung 2}

\vspace{0.5cm}

\begin{tabular}{ b{13.5cm} b{3cm} }
Soll die Piratenpartei Schweiz soll die gefasste Parole nicht vertreten? (Stimmfreigabe)
& \_\_\_\_\_\_\_\_\_\_\_\_\_\_\_\_ \\
\end{tabular}

\vspace{0.5cm}

\begin{tabular}{ b{13.5cm} b{3cm} }
Soll die Piratenpartei Schweiz soll die gefasste Parole im Falle einer Ja-Parole vertreten?
& \_\_\_\_\_\_\_\_\_\_\_\_\_\_\_\_ \\
\end{tabular}

\vspace{0.5cm}

\begin{tabular}{ b{13.5cm} b{3cm} }
Soll die Piratenpartei Schweiz soll die gefasste Parole im Falle einer Nein-Parole vertreten?
& \_\_\_\_\_\_\_\_\_\_\_\_\_\_\_\_ \\
\end{tabular}

\vspace{0.5cm}

Den Stimmzettel bitte nicht markieren!

\end{minipage}


}

\end{document}

